\documentclass[a4paper,12pt]{article}
\usepackage[MeX]{polski}
\usepackage[utf8]{inputenc}

%opening
\title{Matematyka w LaTeX'u}
\author{Michał Kozłowski}

\begin{document}

\maketitle

\begin{abstract}

Ułamek w tekście $ \frac{1}{x} $\\
Oto równanie $c^{2} = a^{2}+b^{2}$\\

Ułamek w tekście $$ \frac{1}{x} $$
Oto równanie $$c^{2} = a^{2}+b^{2}$$
\\

Ułamek
\begin{equation}
\frac{1}{x}
\label{eq:rownanie1}
\end{equation}

Oto równanie

\begin{equation}
$$c^{2}=a^{2}+b^{2}$$
\label{eq:rownanie2}
\end{equation}

Zad. 1.
\eqref{eq:rownanie1}
\eqref{eq:rownanie2}


\\
Zad. 2.\\
Indeks górny
$$ x^y \ e^{x} \ 2^{e} \ A^{2 \times 2} $$\\
Indeks dolny
$$ x_y \ a_{ij} $$
Przykłady:\\
$$ \frac{2^k}{2^{k+2}} $$
$$ 2^{\frac{x^2}{(x+2)(x-2)^3}	$$
$$ \vec{x} = [x_1,x_2, \dots, x_N]	$$

Operatory mat.:
$$ \sum \ \sum_{i=1}^{10}x_{i} \ \prod \ \coprod \ \int \ \oint \ \bigcap \ \bigcup \
	\bigsqcup \ \bigvee \ \bigwedge \ \bigdot \ \bigotimes \ \bigoplus \ \biguplus
$$
Znaczki:
$$ \hat{a} \ \check{b} \ \breve{c} \ \acute{d} \ \grave{e} \ \tilde{f} \ \bar{g} \ 
	\vec{h} \ \dot{m} \ \ddot{n}
$$
Inne:
$$ \widetilde{aaa} \ \widehat{bbb} \ \overleftarrow{ccc} \ \overrightarrow{ddd} \
	\overline{eee} \ \overbrace{fff} \ \underbrace{ggg} \ \underline{hhh} \
	\sqrt{iii} \sqrt[n]{jjj} \ \frac{kkk}{}
$$

Symbole\\

$ \aleph \ \hbar \ \imath $ \dots 

$ \flat \ \natural \ \sharp \ \| \clubsuit \ \S \ \vdots $ \dots 


Formatowanie w trybie mat.: \\

$$	\cmph{Pochyle} \ \textbf{Grube}$$
$$ \textsc{Tekst} \ \verb"Maszynowe"	$$

\\ Reszta punktu 1.:

$ ( \ [ \ \{ \ \downarrow $
$ < \ \geq \ \simeq \ \models \ \approx \ \times \ \circ \ \wedge \ \star $ \\
(i wiele innych)

Równanie:
$$
		54) e'_{ij} = \left \{ \begin{array}{c}
													e_{ij} \ {\rm gdy} \ d(x_i) \neq d(x_j) \\
													a + b = c
											\end{array}
											\right
$$

Algorithmic:
$$
\begin{algorithmic}
%\STATE{ Zalozenie jakies}
\IF{a + b = c}
%\item{b = c - a}
\ENDIF
\end{algorithmic}
$$


\end{abstract}

\section{}

\end{document}