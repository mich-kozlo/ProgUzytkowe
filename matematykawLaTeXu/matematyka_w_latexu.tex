\documentclass[a4paper,12pt]{article}
\usepackage[MeX]{polski}
\usepackage[utf8]{inputenc}
\usepackage{algpseudocode}

%opening
\title{Matematyka w LaTeX'u}
\author{Michał Kozłowski}

\begin{document}

\maketitle

\section{Ćwiczenia}

\TeX wymawiamy jako $\tau\epsilon\chi$.\\
100m$^3$ wody

Ułamek:\\
Jeden ,,\$'' -- ułamek w tekście $ \frac{1}{x} $.\\
Oto równanie $c^{2} = a^{2}+b^{2}$\\

Dwa dolary: ,,\$\$'' -- ułamek na środku: $$ \frac{1}{x} $$
Równanie $$c^{2} = a^{2}+b^{2}$$

Równania:
Równanie z numeracją:
\begin{equation}
\frac{1}{x}
\label{eq:rownanie1}
\end{equation}

Oto równanie:
\begin{equation}
c^{2}=a^{2}+b^{2}
\label{eq:rownanie2}
\end{equation}

Bez numerowania, sam wzór:
\begin{displaymath}
	(a+b)\cdot c=ac+bc
\end{displaymath}

Zad. 1.\\
Ułamek ma numer (\ref{eq:rownanie1}), równanie ma numer
(\ref{eq:rownanie2}).

Indeksy:
Zad. 2.\\
Indeks górny:
$$ x^y \ e^{x} \ 2^{e} \ A^{2 \times 2} $$
Indeks dolny:
$$ x_y \ a_{ij} $$
Przykłady:
$$ \frac{2^k}{2^{k+2}} $$
$$ 2^\frac{x^2}{(x+2)(x-2)^3}	$$
$$ \vec{x} = [x_1,x_2, \dots x_N]	$$

Operatory matematyczne:
$$ \sum \ \sum_{i=1}^{10}x_{i} \ \prod \ \coprod \ \int \ \oint \ \bigcap \ \bigcup \
	\bigsqcup \ \bigvee \ \bigwedge \ \bigodot \ \bigotimes \ \bigoplus \ \biguplus $$
	
,,Ozdobniki'' liter:\\
$$ \hat{a} \ \check{b} \ \breve{c} \ \acute{d} \ \grave{e} \ \tilde{f} \ \bar{g} \ 
	\vec{h} \ \dot{m} \ \ddot{n}$$
	
Strzałki, podkreślenia, klamry, pierwiastki:\\
$$ \widetilde{aaa} \ \widehat{bbb} \ \overleftarrow{ccc} \ \overrightarrow{ddd} \
	\overline{eee} \ \overbrace{fff} \ \underbrace{ggg} \ \underline{hhh} \
	\sqrt{iii} \sqrt[n]{jjj} \ \frac{kkk}{}$$

Symbole:\\
$ \aleph \ \hbar \ \imath $ \dots 
$ \flat \ \natural \ \sharp \ \| \clubsuit \ \S \ \vdots $ \dots 


Formatowanie w trybie mat.: \\
$$	\emph{Pochyle} \ \textbf{Grube}$$
$$ \textsc{Tekst} \ \verb"Maszyna"	$$

Jeszcze inne znaki:
$ ( \ [ \ \{ \ \downarrow $
$ < \ \geq \ \simeq \ \models \ \approx \ \times \ \circ \ \wedge \ \star$
(i wiele innych\dots)

Układ równań:
$$54) e'_{ij}= 
\left \{ 
			\begin{array}{c}
			e_{ij} \ {\rm gdy}\ d(x_i) \neq d(x_j) \\
			a + b = c\\
\end{array}
\right. $$

Algorytm:
\begin{algorithmic}
\State{Input data}
\If{a + b = c}
\item{b = c - a}
\EndIf
\end{algorithmic}

\subsection{Przykłady wzorów}

\begin{displaymath}
(a_1=a_1(x)) \wedge (a_2=a_2(x)) \wedge \dots \wedge
(a_k=a_k(x)) \Rightarrow (d=d(u))
\end{displaymath}

\begin{displaymath}
	[x]_A=\{y\in{}U:a(x)=a(y),\forall{}a\in{}A\}
	\textrm{, where the central
	object }x\in{}U
\end{displaymath}

\begin{displaymath}
	g(u,r)=\{v\in{}U:\frac{card\{IND(u,v)\}}
	{card\{A\}|}\geq{}r\}
\end{displaymath}

\begin{displaymath}
	\textrm{where, }IND(u,v)=\{a\in{}A:a(u)=a(v)\}
\end{displaymath}

\begin{displaymath}
	T:[0,1]\times[0,1]\rightarrow[0,1],
\end{displaymath}

\begin{displaymath}
	x\Rightarrow_Ty\geq{}r\textrm{ if and only if }T(x,r)\leq{}y
\end{displaymath}

\begin{displaymath}
	x\Rightarrow_Ty = max \{ r:T(x,r)\leq{}y \}
\end{displaymath}

\begin{displaymath}
	\mu_T(x,y,r) \textrm{ if and only if } x\Rightarrow_T y \geq r
\end{displaymath}

\begin{displaymath}
	dis_\varepsilon (u,v) = \frac{ | \{ a\in A: ||a(u) - a(v) || 
	\geq \varepsilon \} | }
	{ |A| }
\end{displaymath}

\begin{displaymath}
	dis_\varepsilon (u,v) = \frac{ | \{ a\in A: ||a(u) - a(v) || 
	< \varepsilon \} | }
	{ |A| }
\end{displaymath}

\begin{displaymath}
	Param(v_d) = \sum_{ \{ v\in U_{trn}:d(v)=v_d  \} } w(v,u,\varepsilon)
\end{displaymath}

\begin{displaymath}
	Param(v_d) = \sum_{ \{ v_p\in U_{trn}:d(v_p)=v_d  \} } w(u_q, v_p),
\end{displaymath}

\begin{displaymath}
	S^{c_i}(a)=\frac{ (\overline{C}^a_i - \hat{C}^a_i )^2 }
	{ Z_{{\overline{C}^{a^2}_i} } + Z_{\hat{C}^{a^2}_i } }, a\in A.
\end{displaymath}

\begin{displaymath}
	C^a_i= \{ a(u):u\in U \textrm{ and } d(u)=c_i\}.
\end{displaymath}

\begin{displaymath}
	F_{c_i}(a)= \frac{MSTR_{c_i}(a)}{MSE_{c_i}(a)}
\end{displaymath}

\begin{displaymath}
	MSTR_{C_i}(a) = card\{ C^a_i \} \ast ( \bar{C}^a_i - \hat{C}^a_i )^2
\end{displaymath}

\begin{displaymath}
	A_{C_i}(a) = C^a_i \wedge_\varepsilon \{U\backslash C^a_i\}
\end{displaymath}

\begin{displaymath}
	\frac{card\{ a(u) \in C^a_i: \frac{|a(u)-\hat{C}^a_i}{train_a}>
		\varepsilon  \}}
	{card\{ C^a_i \}}
\end{displaymath}

\end{document}