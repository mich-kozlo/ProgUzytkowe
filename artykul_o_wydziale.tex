\documentclass[a4paper,12pt]{article}
\usepackage[MeX]{polski}
\usepackage[utf8]{inputenc}

%opening
\title{Wydział Matematyki i Informatyki Uniwersytetu
Warmińsko-Mazurskiego}
\author{Michał Kozłowski}

\begin{document}

\maketitle

\begin{abstract}

Wydział Matematyki i Informatyki Uniwersytetu Warmińsko-Mazurskiego (WMiI) – wydział
Uniwersytetu Warmińsko-Mazurskiego w Olsztynie oferujący studia na dwóch kierunkach:
\begin{itemize}
\item Informatyka
\item Matematyka
\end{itemize}
w trybie studiów stacjonarnych i niestacjonarnych. Ponadto oferuje studia podyplomowe.
Wydział zatrudnia 8 profesorów, 14 doktorów habilitowanych, 53 doktorów i 28 magistrów.

Spis treści
\tableofcontents
\item Misja
\item Opis kierunków[1]
\item Struktura organizacyjna
\item Władze Wydziału
\item Historia Wydziału
\item Nowa siedziba Wydziału
\item Adres
\item Przypisy
\

Misja
Misją Wydziału jest:
Kształcenie matematyków zdolnych do udziału w rozwijaniu matematyki i jej stosowania w innych
działach wiedzy i w praktyce;
Kształcenie nauczycieli matematyki, nauczycieli matematyki z fizyką a także nauczycieli informatyki;
Kształcenie profesjonalnych informatyków dla potrzeb gospodarki, administracji, szkolnictwa oraz życia
społecznego;
Nauczanie matematyki i jej działów specjalnych jak statystyka matematyczna, ekonometria,
biomatematyka, ekologia matematyczna, metody numeryczne; fizyki a w razie potrzeby i podstaw
informatyki na wszystkich wydziałach UWM.

Opis kierunków[1]
Na kierunku Informatyka prowadzone są studia stacjonarne i niestacjonarne:
studia pierwszego stopnia – inżynierskie (7 sem.), sp. inżynieria systemów informatycznych, informatyka
ogólna
studia drugiego stopnia – magisterskie (4 sem.), sp. techniki multimedialne, projektowanie systemów
informatycznych i sieci komputerowych
Na kierunku Matematyka prowadzone są studia stacjonarne:
studia pierwszego stopnia – licencjackie (6 sem.), sp. nauczanie matematyki, matematyka stosowana
studia drugiego stopnia – magisterskie (4 sem.), sp. nauczanie matematyki, matematyka stosowana
oraz studia niestacjonarne:
studia drugiego stopnia – magisterskie (4 sem.), sp. nauczanie matematyki
Państwowa Komisja Akredytacyjna w dniu 19 marca 2009r. oceniła pozytywnie jakość kształcenia na kierunku
Matematyka, natomiast w dniu 12 marca 2015r. oceniła pozytywnie jakość kształcenia na kierunku
Informatyka [2].

Struktura organizacyjna
Katedry:
Katedra Algebry i Geometrii
Katedra Analizy i Równań Różniczkowych
Katedra Analizy Zespolonej
Katedra Fizyki i Metod Komputerowych
Katedra Fizyki Relatywistycznej
Katedra Informatyki i Badań Operacyjnych
Katedra Matematyki Dyskretnej i Teoretycznych Podstaw Informatyki
Katedra Matematyki Stosowanej
Katedra Metod Matematycznych Informatyki
Katedra Multimediów i Grafiki Komputerowej
Ośrodki:
Ośrodek Informatyczno-Sieciowy
Władze Wydziału
Dziekan i prodziekani na kadencję 2016-2020:
Dziekan: dr hab. Jan Jakóbowski, prof. UWM
Prodziekan ds. nauki: prof. dr hab. Aleksy Tralle, prof. zw.
Prodziekan ds. studenckich: dr Aleksandra Kiślak-Malinowska
Prodziekan ds. kształcenia: dr Piotr Artiemjew
Historia Wydziału
Wydział Matematyki i Informatyki został utworzony 1 września 2001 roku, po powołaniu dwa lata wcześniej
Uniwersytetu Warmińsko-Mazurskiego, ale jego korzenie sięgają lat 50. XX w. Decyzję o powołaniu Wydziału
podjął Senat UWM w dniu 10 lipca 2001 r. Badania związane z zastosowaniami matematyki rozpoczęły się
wraz z powołaniem w 1950 roku Zakładu Matematyki w Zespołowej Katedrze Fizyki, a od 1951 roku –
Katedry Statystyki Matematycznej ówczesnej Wyższej Szkoły Rolniczej, przemianowanej w 1972 r. na
Akademię Rolniczo Techniczną. Natomiast kształcenie matematyczne i badania w dziedzinie matematyki
zapoczątkowane zostały wraz z utworzeniem w roku 1969 Wyższej Szkoły Nauczycielskiej (od 1974 r. pod
nazwą Wyższa Szkoła Pedagogiczna). Wydział jest kontynuatorem działań Katedry Zastosowań Matematyki
ART oraz Instytutu Matematyki i Fizyki WSP.
Od 27 kwietnia 2009 Wydziałowi przyznano prawo do nadawania stopnia naukowego doktora w dziedzinie
nauk matematycznych w dyscyplinie matematyka[3]
.
Nowa siedziba Wydziału
Uniwersytet Warmińsko-Mazurski w Olsztynie w lipcu 2009 roku podpisał umowę z Polską Agencja Rozwoju
Przedsiębiorczości w ramach projektu "Udoskonalenie infrastruktury i wyposażenia laboratoryjnego nauk
technicznych i informatycznych". Inwestycja na ponad 96 mln złotych realizowana jest ze środków Programu
Operacyjnego Rozwoju Polski Wschodniej 2007-2013. W ramach przedsięwzięcia całkowicie od podstaw
zbudowane zostały obiekty należące do Regionalnego Centrum Informatycznego (RCI).
Kompleks RCI jest wizytówką olsztyńskiego Uniwersytetu - ośrodkiem nowoczesnych technologii IT. Jego
lokalizacja na niewielkim wzniesieniu sprawia że, jest on pierwszym obiektem widocznym zaraz po wjeździe
do Olsztyna od strony Warszawy. W ramach projektu powstało 5 obiektów o różnej liczbie kondygnacji (1-4) i
powierzchni użytkowej 8047 m2 oraz kubaturze 45898,58 m3, połączonych ze sobą przeszklonym foyer. W
budynkach jednorazowo może przebywać około 1200 osób. Obok budynku znajduje się parking na 220
samochodów.
Projekt nowej siedziby Wydziału opracowała sp. z o.o. GENERAL-PROJEKT z Olsztyna pod kierunkiem mgr
inż. Anny Urban. Wykonawcą prac budowlanych była firma Skanska S.A. Prace budowlane zostały
zakończone w czerwcu 2011 roku i od początku roku akademicki 2011/2012 zajęcia dydaktyczne prowadzone
są w nowym budynku.
Budynek RCI użytkowany jest przez dwie instytucje:
Wydział Matematyki i Informatyki
Centrum Zarządzania Infrastrukturą Teleinformatyczną, w skład którego wchodzi Serwerownia UW-M,
Ośrodek obliczeniowy, regionalny węzeł krajowej sieci Pionier, Ośrodek Zarządzania i Eksploatacji
Miejskiej Sieci Komputerowej OLMAN.
W zasoby Wydziału wchodzi ponad 20 pracowni i laboratoriów informatycznych i fizycznych przeznaczonych
do dydaktyki i badań, duża aula pozwalająca zmieścić około 250 osób, dwie mniejsze aule po około 130 osób
każda, 10 sal seminaryjnych, duża sala seminaryjna, pomieszczenia dziekanatu oraz liczne pokoje pracownicze.
Natomiast RCI ma do dyspozycji pracownię podstaw informatyki, bardzo często potrzebną do prowadzenia
szkoleń i warsztatów dla kadry Uczelni oraz salę konferencyjną wraz z terminalem do wideokonferencji.
W ramach kompleksu RCI Wydział Matematyki i Informatyki wzbogacił swoją bazę dydaktyczną o
nowoczesne pracownie i laboratoria, w szczególności:
Laboratoria informatyczne:
Laboratorium architektury komputerów i sieci teleinformatycznych,
Laboratorium technik multimedialnych,
Laboratorium języków programowania,
Laboratorium systemów operacyjnych,
Laboratorium systemów informatycznych,
Laboratorium sztucznej inteligencji i robotyki,
Laboratorium systemów wbudowanych,
Laboratorium systemów mobilnych,
Laboratorium technologii sieciowych,
Laboratorium wspomagania projektowania.
Pracownie fizyczne i elektroniczne:
Dwie pracownie fizyki ogólnej,
Pracownia fizyki technicznej,
Pracownia elektroniki i techniki pomiarowej,
Pracownia elektroniczna - techniki cyfrowej i transmisji sygnałów.
Pracownie naukowe:
Pracownia transmisji i przetwarzania multimediów,
Pracownia symulacji komputerowych i rzeczywistości wirtualnej,
Pracownia komputerowych systemów medycznych (rozpoznawania obrazów),
Pracownia robotyki inteligentnej,
Pracownia nanotechnologii,
Pracownia metod spektroskopowych.
Dzięki środkom uzyskanym z UE pracownie i laboratoria wyposażone są w najnowsze urządzenia światowej
klasy producentów. Możemy do nich zaliczyć:
Najnowsze urządzenia typu Pocket PC a także w pełni wyposażony zestaw sieci teleinformatycznej
nastawionej na komunikację VoIP.
Sterowniki PLC wraz z programatorami i wszelkimi niezbędnymi akcesoriami oraz płyty startowe FPGA
to wyposażenie laboratorium systemów wbudowanych.
Zestawy do konfiguracji sieci informatycznej działającej w trybie komunikacji przewodowej jak i
bezprzewodowej.
Układy do badania zjawisk fizycznych, np. efekt Comptona, efekt Kerra, efekt termoelektryczny, badania
energii promieniowania (alfa), badania pompowania optycznego, promieniowania X, eksperymentu
Sterna-Gerlacha, pomiaru prędkości światła i wielu innych.
Zestaw inteligentnych robotów humanoidalnych oraz zestaw uniwersalnych robotów programowalnych.
Jedną z największych inwestycji był zakup wyposażenia do pracowni nanotechnologii oraz pracowni metod
spektroskopowych. W pracowni nanotechnologii zainstalowany jest mikroskop sił atomowych (AFM) co
pozwala na badanie powierzchni materiałów w tym ferromagnetycznych, izolatorów oraz badania właściwości
żywych komórek w ich naturalnym ciekłym stanie. Dużą zaletą mikroskopu AFM jest możliwość badania
próbek w powietrzu, cieczy i próżni. Z kolei pracownia metod spektroskopowych wyposażona jest w
urządzenie do napylania ultra cienkich warstw i spektrometr (FTIR).
W obrębie budynku, zwłaszcza w pracowniach i laboratoriach, znajduje się około 250 komputerów klasy PC.
Nowy budynek to nie tylko laboratoria czy sale seminaryjne. To również Akademicki Ośrodek Obliczeniowy.
Zbudowany w oparciu o technologię klastrowania serwerów może hostowć wiele serwerów wirtualnych
obsługujących edukację oraz potrzeby administracji Uczelni. Osobna grupa odpowiedzialna jest za
wirtualizację desktopów z preinstalowanym specjalistycznym oprogramowaniem dla naukowców. Kolejna pula
to dynamicznie tworzone grupy takich samych zestawów maszyn wirtualnych do laboratoriów
komputerowych. Dodatkowo wykorzystana jest wirtualizacja aplikacji. Zalety takiego rozwiązania to:
Wirtualne desktopy pracują na wydajnych serwerach; można z nich korzystać na słabym i starym
sprzęcie, który musi tylko uruchomić połączenie z klastrem;
Te same silne komputery są dostępne w każdej pracowni w obrębie kampusu, więc prowadzący nie są
uzależnieni od dostępu do konkretnej pracowni z danym oprogramowaniem specjalistycznym;
Dzięki dynamicznemu tworzeniu komputerów wirtualnych z szablonów, do każdego specjalistycznego
programu jest odrębny szablon desktopu; w ten sposób oszczędzamy zasoby licencji do niezbędnego
minimum pozwalając na równoległą pracę w tym samym czasie, ale w różnych pracowniach;
To samo dotyczy oprogramowania do badań naukowych, które jest dostępne na całej uczelni dla każdego
naukowca;
Dynamiczne tworzenie i zamykanie środowiska zabezpiecza przed rozprzestrzenianiem wirusów oraz
żmudnym czyszczeniem systemów operacyjnych z plików pozostawianych przez studentów.
Do przechowywania systemów wirtualnych wykorzystana jest macierz z wbudowanymi serwerami NAS.
Podstawą są dyski pracujące w technologii Fibre Channel oraz dodatkowo zestaw dysków SATA. Taka
konfiguracja umożliwia szybką obsługę serwerów wirtualizacji, a dyski SATA pozwalają na tworzenie dysków
współdzielonych dla systemów Windows i Linux.
Bez dostępu do światowej sieci Internet, Uniwersytet nie mógłby w pełni realizować swojej misji. Nowy
kompleks to również dodatkowy węzeł sieciowy, który zapewnia uczelni nadmiarowość w dostępie do
szkieletu sieci Pionier i obsługi sieci Uczelni. Nad ciągłością działania infrastruktury sieciowej i serwerowej
czuwa z kolei awaryjny system zasilania składający się z zasilaczy bateryjnych podtrzymujących zasilanie do
chwili uruchomienia agregatu prądotwórczego.
Adres
Wydział Matematyki i Informatyki
ul. Słoneczna 54
10-710 Olsztyn
\begin{thebibliography}{3}
\bibitem{1} Wydział Matematyki i Informatyki UWM w Olsztynie, Oferta kształcenia (http://wmii.uwm.edu.pl/wydzial/oferta-ksztal
cenia). [dostęp 2016-10-12].
\bibitem{2} Wydział Matematyki i Informatyki UWM w Olsztynie, Akredytacja (http://wmii.uwm.edu.pl/wydzial/akredytacja) .
[dostęp 2016-10-13].
\bibitem{3} :: Wydział Matematyki i Informatyki UWM w Olsztynie :: (http://wmii.uwm.edu.pl/index.php?content=uprawnienia)
\end{thebibliography}


\end{abstract}

\section{}

\end{document}